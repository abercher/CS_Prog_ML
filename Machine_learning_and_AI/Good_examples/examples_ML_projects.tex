
%% article for article size, book for book size documents
\documentclass[11pt,a4paper]{article} % use larger type; default would be 10pt

%% Packages

\usepackage[utf8]{inputenc} % set input encoding 
%\usepackage[francais]{babel} %accens aigus acceptés dans le texte
\usepackage{hyperref} % for web links
\usepackage{amssymb, amsmath, amsfonts} % symbols math (\mathbb etc...).
\usepackage{geometry}
\usepackage{fancyhdr} % This should be set AFTER setting up the page geometry
\usepackage{enumerate} % to be able to determine the style of the enumeration
\usepackage[toc,page]{appendix} % For appendices
\usepackage[final]{pdfpages} % to include a pdf

%% Page dimensions, landscape format

\geometry{margin=1in}
%\geometry{legalpaper, landscape, margin=2in}


%% Headers and footers

\pagestyle{fancy} % options: empty , plain , fancy
\renewcommand{\headrulewidth}{1pt} % customise the layout...
\lhead{}\chead{}\rhead{Aritz Bercher}
\lfoot{}\cfoot{\thepage}\rfoot{}

%% Personal shortcuts

\renewcommand{\epsilon}{\varepsilon}%%%%%% pour redéfinir une commande qui existe déjà
\newcommand{\R}{\mathbb{R}}
\newcommand{\C}{\mathbb{C}}
\newcommand{\N}{\mathbb{N}}
\newcommand{\Z}{\mathbb{Z}}
\newcommand{\Q}{\mathbb{Q}}

%% Title

\title{Latex Template for mathematics}
\author{Aritz Bercher}
\date{\today}


\begin{document}

\maketitle

\begin{abstract}
I will try to gather here some good examples of notebooks or codes for machine learning projects that I found, which could help me in the future.
\end{abstract}

\section{Data exploration/visualisation}
There is a good kernel on Kaggle which shows how to correlates different features with a binary variable here:\\
\url{https://www.kaggle.com/startupsci/titanic-data-science-solutions}

\section{Computer vision, Image recognition}

\subsection{Binary classification}
The cats vs dogs example of the course fast ai (see notebook1) is good, although it uses only the tools of the fastai library. I have it on my computer (11.02.18).

\subsection{Multilabel classification}
The example coming from the Kaggle Planet Competition (\href{https://www.kaggle.com/c/planet-understanding-the-amazon-from-space}{kaggle: planet understanding the amazon from space}) is a good example. There is a notebook made available by Kaggle to get started with it (\href{https://www.kaggle.com/robinkraft/getting-started-with-the-data-now-with-docs}{found here}). It shows how to
\begin{enumerate}
\item manipulate the data initially provided as a .csv table, transform the label into binary variables
\item print coocurence matrix
\item normalize the image channels to a reference color curve (change mean and variance of the pixel to match some value defined by the mean and variance of the pixels of already treated pictures).
\end{enumerate}



\end{document} 



