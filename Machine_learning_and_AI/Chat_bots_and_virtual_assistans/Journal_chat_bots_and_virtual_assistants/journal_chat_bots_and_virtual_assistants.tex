

%% See "book", "report", "letter" for other types of document.
\documentclass[11pt,a4paper]{article}
%\documentclass[11pt,a4paper]{article} % use larger type; default would be 10pt

%% Packages
\usepackage{lipsum} % to get some dummy text
\usepackage[utf8]{inputenc} % set input encoding 
%\usepackage[francais]{babel} %accens aigus acceptés dans le texte
\usepackage{hyperref} % for web links
\usepackage{amssymb, amsmath, amsfonts} % symbols math (\mathbb etc...).
\usepackage{geometry} % for page dimensions, landscape, format
\usepackage{fancyhdr}
\usepackage{enumerate} % to be able to determine the style of the enumeration
%\usepackage[toc,page]{appendix} % For appendices
%\usepackage[final]{pdfpages} % to include a pdf

%% Definition of the environment for a diary
\newenvironment{loggentry}[2]% date, heading
{\noindent\textbf{#1}\hspace{1cm}$\mathbf{\sim}$\text{ }\textbf{#2}\\}{\vspace{0.5cm}}


%% Page dimensions, landscape format
%\geometry{legalpaper, margin=1in} % c.f. doc package geometry


%% Headers and footers
%% This should be set AFTER setting up the page geometry
\pagestyle{fancy} % options: empty , plain , fancy
\renewcommand{\headrulewidth}{1pt} % customise the layout...
\lhead{}\chead{}\rhead{Aritz Bercher}
\lfoot{}\cfoot{\thepage}\rfoot{}

%% Title

\title{Chat bots and virtual assistants\\ Journal}
\author{Aritz Bercher}
\date{\today}

\begin{document}

\maketitle

\begin{abstract}
I find the topic of chat bots and virtual assistants really cool. I will try to gather here some information found on the web.
\end{abstract}

\section{Journal}

\begin{loggentry}{30.04.18}{Chat bots: Tai, Xiaoice, Named entity recognition, A nice blog about NLP}
It seems that some fairly advanced chatbots already exist:\\
\href{https://en.wikipedia.org/wiki/Xiaoice}{Wiki: Xiaoice}\\
\href{https://en.wikipedia.org/wiki/Tay_(bot)}{Wiki: Tay}\\
\end{loggentry}

\begin{loggentry}{01.05.18}{Google assistant, Dialogflow (API.AI), some french companies making chatbots}
I discovered the company \textbf{Dialogflow} owned by google:\\
\href{https://en.wikipedia.org/wiki/Dialogflow}{Wiki: Dialogflow}\\
and also read about \textbf{Google assitant}:\\
\href{https://en.wikipedia.org/wiki/Google_Assistant}{Wiki: Goolge Assitant}\\
I also discovered the following french companies/start-ups specialized in chatbots:\\
\begin{itemize}
\item Zelros: \url{http://www.zelros.com/}
\item recast ai: \url{https://recast.ai/}
\item golembot: \url{http://golembot.net/}
\end{itemize}
\end{loggentry}

\begin{loggentry}{02.05.18}{A Coursera course to build your own bot}
Roman recommended this course about NLP:\\
\url{https://www.coursera.org/learn/language-processing}\\
In the description of the course, they say that the final project consist in building your own chatbot.
\end{loggentry}

\begin{loggentry}{09.01.19}{5 levels of Chat bots}

This page explains what are the 5 levels of chat bots. For now we are currently reaching level 3 (Google is at least):\\
\url{https://www.oreilly.com/ideas/the-next-generation-of-ai-assistants-in-enterprise}\\

\end{loggentry}

\begin{loggentry}{01.01.19}{Conversation One: The anatomy of a modern conversation application}

Vijeta shared this link which gives an idea of concepts like \textbf{ontology} (which from my understanding comes down to the definition of intents and entities related to the domain), and how to implement a DM using the context:

\url{https://conversation.one/2017/10/25/anatomy-modern-conversational-application/}

\end{loggentry}


\begin{loggentry}{06.01.19}{Rasa Core and Interactive learning}

In this article:\\
\url{https://blog.rasa.com/a-new-approach-to-conversational-software/}\\
it is explained that Rasa developped a way to create a DM using ``interractive learning" which is a kind of reinforcement learning with feedback at every message, to build a good probabilistic model. I guess that it is good for the intent classification. It seems to be quite easy to use.

\end{loggentry}

\end{document}


















































