

%% See "book", "report", "letter" for other types of document.
\documentclass[11pt,a4paper]{article}
%\documentclass[11pt,a4paper]{article} % use larger type; default would be 10pt

%% Packages
\usepackage{lipsum} % to get some dummy text
\usepackage[utf8]{inputenc} % set input encoding 
%\usepackage[francais]{babel} %accens aigus acceptés dans le texte
\usepackage{hyperref} % for web links
\usepackage{amssymb, amsmath, amsfonts} % symbols math (\mathbb etc...).
\usepackage{geometry} % for page dimensions, landscape, format
\usepackage{fancyhdr}
\usepackage{enumerate} % to be able to determine the style of the enumeration
%\usepackage[toc,page]{appendix} % For appendices
%\usepackage[final]{pdfpages} % to include a pdf

%% Definition of the environment for a diary
\newenvironment{loggentry}[2]% date, heading
{\noindent\textbf{#1}\hspace{1cm}$\mathbf{\sim}$\text{ }\textbf{#2}\\}{\vspace{0.5cm}}


%% Page dimensions, landscape format
%\geometry{legalpaper, margin=1in} % c.f. doc package geometry


%% Headers and footers
%% This should be set AFTER setting up the page geometry
\pagestyle{fancy} % options: empty , plain , fancy
\renewcommand{\headrulewidth}{1pt} % customise the layout...
\lhead{}\chead{}\rhead{Aritz Bercher}
\lfoot{}\cfoot{\thepage}\rfoot{}

%% Title

\title{Java Journal}
\author{Aritz Bercher}
\date{\today}

\begin{document}

\maketitle

\begin{abstract}
With the Coursera - Princeton Algorithms Part 1 course, I had to start learning Java. I will collect here useful information.
\end{abstract}

\begin{loggentry}{18.05.2021}{Getting started}

Following the advice in the HelloWorld tutorial:\\
\url{https://lift.cs.princeton.edu/java/linux/}\\
I downloaded IntelliJ.\\

Here are a couple of interesting things for Java, when coming from Python:\\
\begin{itemize}
\item There is one class per file and it should have the same name as the script.
\item Imports are done automatically if we are importing something in the same folder.
\item I have installed Java 11:\\
\begin{verbatim}
(base) aritz@aritz-ThinkPad-T460p:~$ javac -version
javac 11.0.11
(base) aritz@aritz-ThinkPad-T460p:~$ java -version
openjdk version "11.0.11" 2021-04-20
OpenJDK Runtime Environment (build 11.0.11+9-Ubuntu-0ubuntu2.20.04)
OpenJDK 64-Bit Server VM (build 11.0.11+9-Ubuntu-0ubuntu2.20.04, mixed mode, sharing)
\end{verbatim}
\end{itemize}

I did this \textbf{beginner tutorial for IntelliJ}:\\
\url{https://www.jetbrains.com/help/idea/creating-and-running-your-first-java-application.html}

\end{loggentry}


\begin{loggentry}{23.05.2021}{How Java works}

I read this explanation of Georgi from the Coursera Algorithms Part 1 forum about \textbf{ClassPath} (how Java imports classes used inside a program):\\
\url{https://www.coursera.org/learn/algorithms-part1/discussions/forums/h8l7dMa0Eeatkg4PhuP1KA/threads/LFNq5AtTSSaTauQLU5kmtQ}\\
There are also these two pages for more details:\\
\url{https://en.wikipedia.org/wiki/Classpath}\\
and\\
\url{https://docs.oracle.com/javase/8/docs/technotes/tools/windows/findingclasses.html#A1012444}\\

I also read this article presenting the \textbf{history of Java}:\\
\url{https://www.coursereport.com/blog/what-is-java-programming-used-for}\\
It is non-technical but it gives a good idea of what Java is. The article mentions this book as a good technical reference book, free to read online:\\
\url{http://math.hws.edu/javanotes/}\\

I watched this video which explains at a very high level \textbf{how Java works}:\\
\url{https://www.youtube.com/watch?v=fhfVkPpIwjk}\\
Here are things I found interesting:\\
\begin{enumerate}
\item \textbf{Java bytecode} isn't machine language. It's a transformed version of the code, created by the \textbf{Compiler} from the source code (blabla.java). This code is then run instruction by instruction by an interpreter. The Java bytecode format is blabla.class.
\item The \textbf{Java Virtual Machine (JVM)}, the part which runs the Java bytecode is machine specific but the Java bytecode is universal.
\end{enumerate}

This tutorial gives another overview of Java with more focus on technical aspects:\\
\url{https://beginnersbook.com/2013/05/java-introduction/}\\
This page gives a nice picture of how \textbf{JVM}, \textbf{JRE}, and \textbf{JDK} are related:\\
\url{https://beginnersbook.com/2013/05/jvm/}

\end{loggentry}


\begin{loggentry}{05.06.2021}{Difference between \texttt{class} and \texttt{interface}}

This page explains well the difference between \texttt{class} and \texttt{interface}:\\
\url{https://www.geeksforgeeks.org/differences-between-interface-and-class-in-java/}

Interfaces appeared on slide 47 of the slides of Week 2 of Coursera Algorithms Part 1.

\end{loggentry}


\end{document}






